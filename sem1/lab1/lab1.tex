\documentclass[12pt]{article}

\usepackage{cmap}
\usepackage[utf8]{inputenc}
\usepackage[X2,T2A]{fontenc}
\usepackage[russian,english]{babel}
\usepackage{array}
%\selectlanguage{russian}
\usepackage[left=10mm, top=20mm, right=10mm, bottom=20mm, nohead, nofoot]{geometry}
\usepackage{graphicx}
\usepackage{mathtools}

\begin{document}

\begin{center}
    \large\textrm{Санкт-Петербургский национальный исследовательский Академический университет имени Ж.И. Алфёрова Российской академии наук}
\end{center}

\noindent\rule{\textwidth}{0.5pt}\\

\begin{minipage}{0.5\textwidth}
  \begin{flushleft}
	\textsc{Группа:} 101.1
	
	\textsc{Студенты:} Ряснянский Евгений, Гетманский Андрей, Гамал Эл Дин Амир, Пухов Евгений
	
	\textsc{Преподаватель:} Ефремова Е. А.
	
	\textsc{Лаборант:} Василькова Е.
  \end{flushleft}
\end{minipage}
\begin{minipage}{0.5\textwidth}
  \begin{flushleft}
	\textsc{К работе допущен:} 12.09.2025 г.
	
	\textsc{Работа выполнена: 12.09.2025 г.}
	
	\textsc{Отчёт принят:}
  \end{flushleft}
\end{minipage}


\begin{center}
     \large\textsc{Рабочий протокол и отчёт по лабораторной работе № 1}
    
    \textbf{\textsc{«Моделирование случайной величины и исследование ее распределения»}}
\end{center}

\noindent\rule{\textwidth}{0.5pt}\\

\begin{enumerate}
    \item \large\textsc{Цель работы}

    Изучение нормального распределения случайных величин.
     
    \item \large\textsc{Объект исследования}

    Случайная величина и её распределение.

    \item \large\textsc{Задачи, решаемые при выполнении работы}
      \begin{itemize}

       \item Провести многократные измерения заданных величин: габариты и массы образцов металлов, сопротивление резисторов.

       
       \item Вычислить среднее значение и дисперсию полученной выборки результатов \\измерений.

       \item Вычислить с учетом погрешности плотности образцов и определить их материал.
     %  \item Построить гистограмму плотности относительной частоты попадания результатов измерения в выбранный интервал времени.

   %    \item Выполнить сравнение гистограммы с нормальным распределением, имеющим те же среднее и дисперсию, что и полученная в эксперименте случайная выборка.

       \item Вычислить случайную и полную погрешности измерения заданных величин для нескольких значений доверительной вероятности.
        \\
      \end{itemize}
      \newpage
    \item \large\textsc{Схема установки}
      
          
       Экспериментальная установка состоит из следующих компонентов:\\
       
       1. Микрометр. \\
       2. Штангенциркуль. \\
       3. Кубики из различных металлов.\\
       4. Мультиметр в режимах измерения сопротивления и напряжения.\\
       5. Резисторы предположительно одного сопротивления.\\
       6. Гальванические элементы предположительно одной ЭДС.\\
      % 4. Металлические шарики.\\
       
    \item \large\textsc{Измерительные приборы}
    \\


     \begin{center}
         
   
      \begin{tabular}{|m{7em}|m{7em}|m{7em}|m{7em}|}
    
       \hline
       Прибор & Тип прибора & Исследуемый диапазон & Погрешность измерения        \\ \hline
       Микрометр & 
       Аналоговый & 0 - 15 мм & 10 мкм\\ \hline
       Мультиметр (измерение напряжения)   & Цифровой & 0 - 10 В & 10 мВ \\ \hline
       Мультиметр (измерение сопротивления)   & Цифровой & 0 - 500 Ом & 0,1 Ом \\ \hline              
      \end{tabular}
     \end{center}   
      \begin{center}
        Таблица 1: Измерительные приборы \\
      \end{center}
    \item \large\textsc{Метод экспериментального исследования}

    Проведение многократных измерений заданных величин.
    

    \item \large\textsc{Формулы}

\text{Среднее значение заданной величины определяется так:}\\
\begin{equation}
    \langle X \rangle = \frac{1}{N} \sum_{i=1}^{N} t_i
\end{equation}
\\

\text{Где $N$ - количество  проведенных измерений}\\
\\
\text{Дисперсия определяется так:}\\
\begin{equation}
    \ D=\sigma^2 
\end{equation}
\begin{equation}
        \sigma = \sqrt{\frac{1}{N} \sum_{i=1}^{N} (X_i-\langle X \rangle)^2}
\end{equation}
\text{Где $\sigma$ - среднеквадратичное отклонение(СКО).}\\
\\
\text{Полная погрешность определяется так:} \\
\begin{equation}
	\Delta x = \sqrt{\Delta x_1^2 + \Delta x_2^2}
\end{equation}
\text{Где $\Delta x_1$ - случайная погрешность, $\Delta x_2$ - погрешность прибора}
    \item \large\textsc{Результаты прямых  измерений и расчёты}
    
    %\begin{table}[!ht]
 \begin{center}
    \begin{tabular}{|m{5em}|m{5em}|m{5em}|m{5em}|m{5em}|m{5em}|}
        \hline
        Номер материала & Номер опыта  & Длина первой грани, мм & Длина второй грани, мм &  Длина третьей грани, мм & Масса, г \\ 
        
        \hline
        1 & 1 & 10.43 & 10.42 & 10.40 & 2.70 \\ \hline
        1 & 2 & 10.43 & 10.40 & 10.39 & 2.71\\ \hline
        1 & 3 & 10.42 & 10.44 & 10.39 & 2.70 \\ \hline 
        2 & 1 & 10.48 & 10.41 & 10.40 & 8.81 \\ \hline
        2 & 2 & 10.49 & 10.40 & 10.40 & 8.81 \\ \hline
        2 & 3 & 10.47 & 10.40 & 10.40 & 8.80 \\ \hline
        3 & 1 & 10.07 & 10.10 & 10.09 & 19.61 \\ \hline
        3 & 2 & 10.08 & 10.08 & 10.07 & 19.60 \\ \hline
        3 & 3 & 10.06 & 10.10 & 10.07 & 19.61 \\ \hline
        4 & 1 & 10.07 & 10.49 & 10.07 & 4.59 \\ \hline
        4 & 2 & 10.07 & 10.49 & 10.07 & 4.59 \\ \hline
        4 & 3 & 10.06 & 10.49 & 10.06 & 4.59 \\ \hline
    \end{tabular}
     \end{center}
    %\end{table}

    \begin{center}
        Таблица 2: Данные прямых измерений 
      \end{center}
\begin{center}
    
    %\begin{table}[!ht]
    \begin{tabular}{|l|l|l|l|}
    \hline
        Номер опыта  & Сопротивление, Ом & Номер опыта  & Сопротивление, Ом \\ \hline
        1 & 385.9 & 16 & 386.5 \\ \hline
        2 & 385.5 & 17 & 386.6 \\ \hline
        3 & 385.8 & 18 & 386.4 \\ \hline
        4 & 386.7 & 19 & 385.8 \\ \hline
        5 & 358.8 & 20 & 386.7 \\ \hline
        6 & 386.7 & 21 & 386.2 \\ \hline
        7 & 386.2 & 22 & 386.6 \\ \hline
        8 & 386.4 & 23 & 386.4 \\ \hline
        9 & 386.2 & 24 & 385.0 \\ \hline
        10 & 386.4 & 25 & 385.6 \\ \hline
        11 & 386.8 & 26 & 385.1 \\ \hline
        12 & 386.6 & 27 & 385.8\\ \hline
        13 & 386.4 & 28 & 385.4\\ \hline
        14 & 386.1 & 29 & 386.2\\ \hline
        15 & 385.9 & 30 & 385.8 \\ \hline
    \end{tabular}
    %\end{table}
\end{center}
    \begin{center}
        Таблица 3: Данные прямых измерений 
       \end{center}

    \item \large\textsc{Вычисления} \\
    \\
    Средние значения длин граней и масс: \\
    $ \langle a_1 \rangle = 10.413 $ мм $ \langle m_1 \rangle = 2.703 $ г \\
    $ \langle a_2 \rangle = 10.428 $ мм $ \langle m_2 \rangle = 8.807 $ г \\
    $ \langle a_3 \rangle = 10.08 $ мм $ \langle m_3 \rangle = 19.607 $ г \\
    $ \langle a_4 \rangle = 10.208 $ мм $ \langle m_4 \rangle = 4.59 $ г \\
    Вычисление погрешности измерений длин граней и масс кубиков: \\ 
    $\Delta a_1 = \sqrt{0.018^2 + 0.01^2} = 0.02$ мм \\
    $\Delta a_2 = \sqrt{0.037^2 + 0.01^2} = 0.039$ мм\\
    $\Delta a_3 = \sqrt{0.013^2 + 0.01^2} = 0.017$ мм\\
    $\Delta a_4 = \sqrt{0.2^2 + 0.01^2} = 0.2$ мм\\
    \\
    $\Delta m_1 = \sqrt{0.005^2 + 0.01^2} = 0.011$ г \\
    $\Delta m_2 = \sqrt{0.005^2 + 0.01^2} = 0.011$ г \\
    $\Delta m_3 = \sqrt{0.005^2 + 0.01^2} = 0.011$ г \\
    $\Delta m_4 = \sqrt{0^2 + 0.01^2} = 0.01$ г \\ \\ \\
    Средние плотности кубиков: \\
    $ \langle \rho_1 \rangle = 2.394 $ г/см$^3$ \\
    $ \langle \rho_2 \rangle = 7.766 $ г/см$^3$ \\
    $ \langle \rho_3 \rangle = 19.144 $ г/см$^3$ \\
    $ \langle \rho_4 \rangle = 4.315 $ г/см$^3$ \\ \\
    Вычисление погрешности значений плотности:\\
    $\Delta \rho_1 = 0.017$ г/см$^3$\\
    $\Delta \rho_2 = 0.088$ г/см$^3$\\
    $\Delta \rho_3 = 0.097$ г/см$^3$\\
    $\Delta \rho_4 = 0.253$ г/см$^3$\\ \\
    Случайная погрешность резисторов: \\
    $\Delta R_1 = 0.48 $ Ом \\ \\
    Полная погрешность резисторов: \\
    $\Delta R = \sqrt{0.48^2 + 0.1^2} = 0.49 $ Ом
    
%    @ \\
%    ЗДЕСЬ ДОЛЖЕН БЫТЬ ОПИСАН ПРОЦЕСС ВЫЧИСЛЕНИЯ \\
%    У КУБИКОВ \\
%    - ПОГРЕШНОСТИ ИЗМЕРЕНИЯ ДЛИН СТОРОН \\
%    - ПЛОТНОСТИ \\
%    - ПОГРЕШНОСТИ ПЛОТНОСТИ \\
%    У РЕЗИСТОРОВ \\
%    - ПОГРЕШНОСТИ \\
%    - СКО \\
%    - ДИСПЕРСИИ \\
%    @
    
    \item \large\textsc{Окончательные результаты} \\
    	Значения плотности у кубиков: \\
		$\rho_1 = 2.394 \pm 0.017$ г/см$^3$\\
		$\rho_2 = 7.766 \pm 0.088$ г/см$^3$\\
		$\rho_3 = 19.144 \pm 0.097$ г/см$^3$\\
		$\rho_4 = 4.315 \pm 0.253$ г/см$^3$\\ 
		
		Сопротивление резисторов: \\
		$ R = 386.12 \pm 0.49 $ Ом
		
		
%    @ \\
%    ЗДЕСЬ ДОЛЖНЫ БЫТЬ УКАЗАНИЯ НА \\
%    У КУБИКОВ \\
%    - ПОГРЕШНОСТЬ ПЛОТНОСТИ \\
%    - АНАЛИЗ ПЛОТНОСТИ И УСТАНОВЛЕНИЕ МЕТАЛЛА \\
%    У РЕЗИСТОРОВ \\
%    - РЕЗУЛЬТАТЫ ВЫЧИСЛЕНИЙ \\
%    @

    \item \large\textsc{Вывод}
\\ В ходе эксперимента было сделано по 3 измерения каждой из граней образцов материалов при помощи точных измерительных приборов, а также было выполнено 30 измерений сопротивлений резисторов с помощью мультиметра в режиме измерения сопротивления. Все данные отражены в таблицах 2 и 3. Для полученных выборок результатов были вычисленны средние значения и погрешности.%, указанные в таблицах ????????? 3 и 5.\par
По результатам измерений были установлены с учетом погрешностей плотности образцов материалов и при анализе таблиц плотности веществ установлены характеры металлов, из которых они сделаны: медь, вольфрам, титан. \par
При анализе результатов измерения сопротивления резисторов было выявлено, что махнимальное значение $\Omega_{max}=0.80085$ и минимальное значение $\Omega_{min}=0.80085$ а дисперсия $\sigma$ составила $8.0085$, что говорит о высоком качестве предоставленных резисторов.\par
В таблицах 8 и 9 частота попадания результатов в доверительные интервалы 0.68, 0.95, 0.99 довольно высока, что свидетельствует о высокой точности в проводимых экспериментах. Заметим также, что  максимумы гистограмм и $\rho_{max}$ отличаются примерно в 2 раза, так как было проведено одинаковое количество измерений и, следовательно, примерно одинакова точность экспериментов.
    
\end{enumerate}
\end{document}
